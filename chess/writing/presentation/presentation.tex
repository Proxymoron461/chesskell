\documentclass{beamer}

\usepackage[utf8]{inputenc}
\usepackage{skak}
\usepackage{listings}
\usepackage{xcolor}
\usepackage{graphicx}

\usepackage[
backend=biber,
style=alphabetic,
citestyle=authortitle-icomp
]{biblatex}

\addbibresource{presentation.bib}

\usetheme{metropolis}

\title{Chesskell: Modelling a Two-Player Game at the Type-Level}
\author{Toby Bailey}
\institute{Department of Computer Science}
\date{\today}

\definecolor{background}{rgb}{0.92, 0.92, 0.92}
\definecolor{comments}{rgb}{0.0, 0.64, 0.0}
\definecolor{keywords}{rgb}{0.0, 0.0, 0.64}
\definecolor{identifiers}{rgb}{0.63, 0.81, 0.94}
\definecolor{strings}{rgb}{1.0, 0.3, 0.0}

\lstset{
    language=haskell,
    basicstyle=\footnotesize\ttfamily,
    backgroundcolor=\color{background},
    keywordstyle=\color{keywords}\bfseries,
    commentstyle=\color{comments}\textit,
    stringstyle=\color{strings},
    % identifierstyle=\color{identifiers},
    breakatwhitespace=true,
    breaklines=true,
    keepspaces=true,
    captionpos=b,
    frame=tlbr,    % Margin at all 4 sides
    framesep=4pt,  % Margin size
    framerule=0pt,
    morekeywords={Eval, Exp, family, instance},
    deletekeywords={map, and, error, take},
    showstringspaces=false
}

% Custom command for all inline code styling
\newcommand{\inline}[1]{\lstinline[basicstyle=\ttfamily]{#1}}

\begin{document}

\frame{\titlepage}

\begin{frame}[standout]

Why do type systems exist?
    
\end{frame}

\begin{frame}{Why do type systems exist?}
Type systems exist because: \pause we want to avoid errors.

(\cite{cardellitypes})

% \pause But not domain-specific logical errors?

% \pause Solution: model your domain in the types!
\end{frame}

\begin{frame}[fragile]{Type Errors}

A type system can prevent certain errors from occurring at all:

\begin{lstlisting}
not 5
\end{lstlisting}

The above will not compile, preventing an error.
    
\end{frame}

\begin{frame}[fragile]{Type Errors cont.}

You have a website, where you sell books.

\pause

For some reason, you use Java to build the server:

\begin{lstlisting}[language=Java]
int noOfPages = -1;
\end{lstlisting}

This is obviously an error. But it compiles!
    
\end{frame}

\begin{frame}[fragile]{Type-Level Programming}

Recent developments to Haskell have focused on performing computation at the type level with \emph{type families} (\cite{opentfs}, \cite{closedtfs}).

Haskell is NOT a dependently typed language; types and values are separated.
    
\end{frame}

\begin{frame}[fragile]{Type Erasure}
In fact, Haskell programs undergo \emph{type erasure}.

\pause

\begin{lstlisting}
x :: Int
x = 3
\end{lstlisting}

Haskell type-level programming involves circumventing type erasure.

\end{frame}

\begin{frame}{Complex Type-Level Computation}

There are other attempts at rule enforcement, in Haskell, at the type level.

\pause

Mezzo is an EDSL for musical composition, which can enforce classical harmony rules at the type level (\cite{mezzohaskellsymposium}).

\pause

BioShake is an EDSL for Bioinformatics workflows, whereby only workflows that are configured correctly will compile (\cite{bioshake}).
    
\end{frame}

\begin{frame}[fragile]{Why create Chesskell?}
These capabilities are intended for use; and so must be tested through use.

What issues do we run into when implementing a complex rule set at the type level?

\pause

\textbf{Is Haskell's type system mature enough for Chess?}

\end{frame}

\begin{frame}{Why Chess?}

\begin{itemize}
    \item<1-3> It's popular and internationally known;
    \item<2-3> It's been widely studied in the field of Computer Science (\cite{chesseducation}, \cite{chessml});
    \item<3-4> It has a \emph{well-defined ruleset}.
\end{itemize}
    
\end{frame}

\begin{frame}{A note on Chess}

A Chess game takes place on a board.

\begin{figure}[h]
    \centering
    \fenboard{8/8/8/8/8/8/8/8 w - - 0 1}
    \showboard
    \label{emptyboard}
\end{figure}

\end{frame}

\begin{frame}{A note on Chess cont.}

There are two \emph{Teams}; Black and White.

\begin{figure}[h]
    \centering
    \newgame
    \showboard
    \label{startboard1}
\end{figure}

\end{frame}

\begin{frame}{A note on Chess cont.}

Each Team has 16 \emph{Pieces}; 8 Pawns, 2 Rooks, 2 Bishops, 2 Knights, a Queen, and a King.
    
\begin{overprint}
    
\onslide<1>\begin{figure}[h]
    \centering
    \fenboard{8/8/8/8/8/8/8/8 w - - 0 1}
    \showboard
    \label{showpiece0}
\end{figure}

\onslide<2>\begin{figure}[h]
    \centering
    \newgame
    \showonly{p,P}
    \showboard
    \label{showpiece1}
\end{figure}

\onslide<3>\begin{figure}[h]
    \centering
    \newgame
    \showonly{p,P,r,R}
    \showboard
    \label{showpiece2}
\end{figure}

\onslide<4>\begin{figure}[h]
    \centering
    \newgame
    \showonly{p,P,r,R,b,B}
    \showboard
    \label{showpiece3}
\end{figure}

\onslide<5>\begin{figure}[h]
    \centering
    \newgame
    \showallbut{q,Q,k,K}
    \showboard
    \label{showpiece4}
\end{figure}

\onslide<6>\begin{figure}[h]
    \centering
    \newgame
    \showallbut{k,K}
    \showboard
    \label{showpiece5}
\end{figure}

\onslide<7>\begin{figure}[h]
    \centering
    \newgame
    \showboard
    \label{showpiece6}
\end{figure}

\end{overprint}
    
\end{frame}

\begin{frame}{A note on Chess cont.}

Each piece has different movement rules, allowing them to move around the 8x8 board.
    
\begin{figure}[h]
    \centering
    \newgame
    \scalebox{0.60}{\showboard}
    \quad
    \hidemoves{1. e4 Nc6}
    \scalebox{0.60}{\showboard}
    \label{demonstratemovement}
\end{figure}
    
\end{frame}

\begin{frame}{A note on Chess cont.}

Pieces can remove other pieces from the board via \emph{capture}; which almost always involves moving to the other piece's square.

\begin{figure}[h]
    \centering
    \fenboard{8/8/8/4Q3/3p4/8/8/8 w - - 0 1}
    \scalebox{0.60}{\showboard}
    \quad
    \hidemoves{1. Qd4}
    \scalebox{0.60}{\showboard}
    \label{demonstratecapture}
\end{figure}

\end{frame}

\begin{frame}[fragile]{A Short Example}

Below is a valid move by a White Pawn:

\begin{figure}[h]
    \centering
    \newgame
    \scalebox{0.55}{\showboard}
    \quad
    \hidemoves{1. e4}
    \scalebox{0.55}{\showboard}
    \label{validpawnmove}
\end{figure}

\pause

\begin{lstlisting}
chess
    pawn e2 to e4
end
\end{lstlisting}

\end{frame}

\begin{frame}[fragile]{A Short Example cont.}

Below is an \emph{invalid} move by a White Pawn:

\begin{figure}[h]
    \centering
    \newgame
    \scalebox{0.55}{\showboard}
    \quad
    \hidemoves{1. e5}
    \scalebox{0.55}{\showboard}
    \label{badpawnmove}
\end{figure}

\pause

\begin{overprint}

\onslide<2>\begin{lstlisting}
chess
    pawn e2 to e5
end
\end{lstlisting}

\onslide<3>\begin{lstlisting}
-- Fails to compile with type error:
--     * There is no valid move from E2 to E5.
--     The Pawn at E2 can move to: E3, E4
chess
    pawn e2 to e5
end
\end{lstlisting}

\end{overprint}

\end{frame}

\begin{frame}[fragile]{A Little Terminology}

In Haskell, values have \emph{types}, and types have \emph{kinds}.

\pause

Luckily, we can \emph{promote} types to kinds with the \inline{-XDataKinds} extension (\cite{givingpromotion}):

\pause

\begin{lstlisting}
data Book = Fiction | NonFiction
\end{lstlisting}

\begin{figure}
    \includegraphics[height=0.6\textheight,keepaspectratio]{Promotion.png}
    \label{promotiondiagram}
\end{figure}

\end{frame}

\begin{frame}[fragile]{A Little Terminology cont.}

In Haskell, you compute on values with \emph{functions}.

\begin{lstlisting}
factorial :: Int -> Int
factorial 0 = 1
factorial x = x * factorial (x - 1)
\end{lstlisting}

\pause

But you have to use type families to compute on types:

\begin{lstlisting}
type family Factorial (x :: Nat) :: Nat where
    Factorial 0 = 1
    Factorial x = Mult x (Factorial (x - 1))
    
type family Mult (x :: Nat) (y :: Nat) :: Nat where
    Mult 0 y = 0
    Mult 1 y = y
    Mult x y = y + (Mult (x - 1) y)
\end{lstlisting}

\end{frame}

\begin{frame}[fragile]{Problems with Type Families?}

Lots of idiomatic Haskell code relies on functions being \emph{first-class}; partial application, mapping, etc.

\begin{lstlisting}
x = map (+ 2) [1,2,3]
-- = [3,4,5]
\end{lstlisting}

\pause

But type families can't be partially applied!

\begin{lstlisting}
-- Type error: type family (+) was expecting 2 arguments, got 1
type X = Map (+ 2) '[1,2,3]
\end{lstlisting}
    
\end{frame}

\begin{frame}[fragile]{Introducing First Class Families}

Thanks to Li-yao Xia, we have First Class Families!

It relies on a data type \inline{Exp}, and a type family \inline{Eval}, to create a type-level interpreter:

\begin{lstlisting}
type Exp a = a -> *
type family Eval (e :: Exp a) :: a
\end{lstlisting}

% You define a new \emph{data type} to hold the arguments, where the return types are wrapped in \inline{Exp}, and an \inline{Eval} instance to define the behaviour of the function.

\end{frame}

\begin{frame}[fragile]{Making a First Class Family}

\begin{lstlisting}
type family And (x :: Bool) (y :: Bool) :: Bool where
    And True  True  = True
    And True  False = False
    And False True  = False
    And False False = False
\end{lstlisting}

becomes:

\begin{lstlisting}
data And :: Bool -> Bool -> Exp Bool
type instance Eval (And True  True)  = True
type instance Eval (And True  False) = False
type instance Eval (And False True)  = False
type instance Eval (And False False) = False
\end{lstlisting}

\end{frame}

\begin{frame}[fragile]{Type-Level Mapping}

With the below definition of \inline{Map}:

\begin{lstlisting}
data Map :: (a -> Exp b) -> f a -> Exp (f b)
type instance Eval (Map f '[])       = '[]
type instance Eval (Map f (x ': xs)) = Eval (f x) ': Eval (Map f xs)
\end{lstlisting}

And a definition of a type-level \inline{(+)}:

\begin{lstlisting}
data (:+) :: Nat -> Nat -> Exp Nat
type instance Eval (Z     :+ y) = y
type instance Eval ((S x) :+ y) = S (x :+ y)
\end{lstlisting}

We can now map over a type-level list:

\begin{lstlisting}
Eval (Map (:+ 2) '[1,2,3])
-- = '[3,4,5]
\end{lstlisting}

\end{frame}

\begin{frame}[fragile]{Representing Movement}

% In Chesskell, we represent movement with a single First Class Family that performs the given movement on a \inline{BoardDecorator}:

Each turn of movement is expressed as a single First Class Family:

\begin{lstlisting}
data Move :: Position -> Position -> BoardDecorator -> Exp BoardDecorator
\end{lstlisting}

\pause

Thanks to First Class Families, we can extend this with rule-checking naturally; using a type-level version of the function composition operator, \inline{(.)}:

\begin{lstlisting}
PostMoveCheck2 . PostMoveCheck1 . Move fromPos toPos . PreMoveCheck2 . PreMoveCheck1
\end{lstlisting}

\end{frame}

\begin{frame}[fragile]{The Board type}

To avoid repeated length checks, we use \emph{length-indexed vectors} with a type-level implementation of Peano natural numbers:

\begin{lstlisting}
data Vec (n :: Nat) (a :: Type) where
    VEnd   :: Vec Z a
    (:->)  :: a -> Vec n a -> Vec (S n) a
\end{lstlisting}

Since a Chess board is always an 8x8 grid, we use vectors of vectors:

\begin{lstlisting}
type Eight = (S (S (S (S (S (S (S (S Z))))))))
type Row   = Vec Eight (Maybe Piece)
type Board = Vec Eight Row
\end{lstlisting}

In the codebase, we use a wrapper data structure (named \inline{BoardDecorator}) to hold additional useful information.

\end{frame}

\begin{frame}[fragile]{Using the Type-Level Model}

To interact with this type level model, the output of each \inline{Move} call is piped to the next one:

\begin{lstlisting}
x = Move a1 a2 StartBoard
y = Move e3 e4 x
z = -- ...
\end{lstlisting}

\end{frame}

\begin{frame}[fragile]{Using the Type-Level Model cont.}

Below is a simplified representation of what happens for the game: \inline{chess pawn a1 to a2 king e2 to e1 end}

\begin{overprint}

\onslide<1>\begin{lstlisting}
(MoveWithCheck King e2 e1 . MoveWithCheck Pawn a1 a2) StartBoard
\end{lstlisting}

\onslide<2>\begin{lstlisting}
(MoveWithCheck King e2 e1 . MoveWithCheck Pawn a1 a2) StartBoard

data MoveWithCheck :: PieceName -> Position -> Position -> Exp Board
type instance Eval (MoveWithCheck name fromPos toPos board)
    -- If there is a piece of that type at fromPos
    = If (IsPieceAt name fromPos board)
        -- then
        (Move fromPos toPos board)
        -- else
        (TypeError -- ...)
\end{lstlisting}

\end{overprint}

\end{frame}

\begin{frame}[fragile]{Interacting with Type-Level model at the value level}

The core idea is wrapping the \inline{BoardDecorator} type in a \inline{Proxy}, so that it can be passed around within a value by functions:

\begin{lstlisting}
data Proxy a = Proxy

edslMove :: SPosition from
         -> SPosition to
         -> Proxy (b :: BoardDecorator)
         -> Proxy (Eval (Move from to b))
edslMove (x :: SPosition from) (y :: SPosition to) (z :: Proxy (b :: BoardDecorator))
    = Proxy @(Eval (Move from to b))
\end{lstlisting}

But this would still look similar to Haskell syntax; we need a new approach.

\end{frame}

\begin{frame}[fragile]{Creating the EDSL}

Ideally, the EDSL should look like existing chess notation:

\begin{verbatim}
1. e4 e5 2. Nf3 Nc6 3. Bb5 a6
\end{verbatim}


% Inspired by the Flat Builders pattern laid out in Mezzo.
Can achieve using Continuation Passing Style, inspired by Dima Szamozvancev's Flat Builders work (\cite{mezzo}).

\end{frame}

\begin{frame}[fragile]{Chess Continuations}

We define a continuation for beginning the stream (named \inline{chess}), another for ending it (\inline{end}), and a series of piece continuations that move a piece of that type; \inline{king}, \inline{queen}, etc.

All of the above continuations can be chained together like so:

\begin{lstlisting}
game = chess pawn a1 to a2 bishop e4 to d5 end
\end{lstlisting}

\end{frame}

\begin{frame}[fragile]{A Longer Example}

Below is a short game, ending in checkmate by White:

\begin{figure}[h]
    \centering
    \newgame
    \scalebox{0.55}{\showboard}
    \quad
    \hidemoves{1. e4 f5 2. Qf3 g5 3. Qh5}
    \scalebox{0.55}{\showboard}
    \label{threemovecheckmate}
\end{figure}

\pause

\begin{lstlisting}
game = chess
    pawn e2 to e4
    pawn f7 to f5
    queen d1 to f3
    pawn g7 to g5
    queen f3 to h5
end
\end{lstlisting}

\end{frame}

\begin{frame}[fragile]{A Longer Example}

What about a piece trying to move after Checkmate, when the game ends?

\begin{overprint}

\onslide<1>\begin{lstlisting}
game = chess
    pawn e2 to e4
    pawn f7 to f5
    queen d1 to f3
    pawn g7 to g5
    queen f3 to h5
    pawn g5 to g4
end
\end{lstlisting}

\onslide<2>\begin{lstlisting}
-- Below results in the following type error:
    -- * The Black King is in check after a Black move. This is not allowed.
    -- * When checking the inferred type
    --     game :: Data.Proxy.Proxy (TypeError ...)
game = chess
    pawn e2 to e4
    pawn f7 to f5
    queen d1 to f3
    pawn g7 to g5
    queen f3 to h5
    pawn g5 to g4
end
\end{lstlisting}

\end{overprint}

\end{frame}

\begin{frame}[fragile]{A Longer Example cont.}

What about if the White Queen tries to move through another piece, mid-game?

\begin{overprint}

\onslide<1>\begin{lstlisting}
game = chess
    pawn e2 to e4
    pawn f7 to f5
    queen d1 to d3  -- Invalid move
    pawn g7 to g5
    queen f3 to h5
end
\end{lstlisting}

\onslide<2>\begin{lstlisting}
-- Below results in the following type error:
    -- * There is no valid move from D1 to D3.
    -- The Queen at D1 can move to: E2, F3, G4, H5, ...
    -- * When checking the inferred type
    -- game :: Data.Proxy.Proxy (...)
game = chess
    pawn e2 to e4
    pawn f7 to f5
    queen d1 to d3  -- Invalid move
    pawn g7 to g5
    queen f3 to h5
end
\end{lstlisting}

\end{overprint}

\end{frame}

\begin{frame}[fragile]{A Longer Short Example}

We also developed a shorthand syntax! 

The below game:

\begin{lstlisting}
game = chess
    pawn e2 to e4
    pawn f7 to f5
    queen d1 to f3
    pawn g7 to g5
    queen f3 to h5
end
\end{lstlisting}

becomes:

\begin{lstlisting}
game = chess
    p e4 p f5
    q f3 p g5
    q h5
end
\end{lstlisting}

\end{frame}

\begin{frame}[standout]

Demo
    
\end{frame}

\begin{frame}{Testing Overview}

Combination of:

\pause

\begin{itemize}
    \item<2-> Unit testing with assertions, based on whether a code snippet compiles or fails to compile;
    \item<3-> Unit tests of custom-made board scenarios, to test out specific behaviour;
    \item<4-> EDSL tests of custom board scenarios, for the same purpose;  
    \item<5-> EDSL testing with famous Chess games, written out in Chesskell notation.
\end{itemize}
    
\end{frame}

\begin{frame}[fragile]{Unit Testing}

Unit tests rely on two main assertions; \inline{shouldTypecheck}, and \inline{shouldNotTypecheck}, which succeed or fail based on whether a specific code snippet fails with a type error or not.

We created unit tests for individual type families, to determine if they have the behaviour they should:

\begin{lstlisting}
oppositeTeamTest1 :: White :~: Eval (OppositeTeam Black)
oppositeTeamTest1 = Refl

-- ...

it "1: OppositeTeam Black = White" $
    shouldTypecheck oppositeTeamTest1
\end{lstlisting}

(Note that a value with type \inline{x :~: y} will only compile if \inline{x} and \inline{y} can be unified.)
    
\end{frame}

\begin{frame}[fragile]{Unit Testing cont.}

We also created unit tests for every FIDE Law of Chess that could be tested in this manner:

\begin{lstlisting}
whiteBishopCannotTakeOwnTeam :: Proxy (a :: BoardDecorator)
whiteBishopCannotTakeOwnTeam = Proxy @(Eval (Move (At C Nat1) (At D Nat2) WhiteStartDec))

-- ...

it "1: A White Bishop cannot take a piece on the same team" $
    shouldNotTypecheck whiteBishopCannotTakeOwnTeam
\end{lstlisting}
    
\end{frame}

\begin{frame}[fragile]{Scenario Testing}
    
We created custom Chess test boards, paired with unit tests, to model specific behaviour: % For example, we created the following board and test, whereby the Black King can castle left, but not right:

\begin{figure}[h]
    \centering
    \fenboard{8/5Q2/8/8/8/8/8/r3k2r b - - 0 1}
    \scalebox{0.55}{\showboard}
    \label{blackcastleleft}
\end{figure}

\begin{lstlisting}
blackCanCastleLeft :: '(True, False) :~: CanCastle Black BlackCastleLeftDec
blackCanCastleLeft = Refl
-- ...
shouldTypecheck blackCanCastleLeft
\end{lstlisting}

\end{frame}

\begin{frame}[fragile]{EDSL Scenario}

The EDSL was similarly tested with scenarios, to ensure that rule-breaking moves did not compile:

\begin{overprint}
    
\onslide<1>\begin{lstlisting}
didntPromoteBlack = create
        put _Wh _P at h7
        put _Bl _P at a2
    startMoves
        pawn h7 promoteTo _B h8
        pawn a2 to a1
    end
\end{lstlisting}

\onslide<2>\begin{lstlisting}
-- Below fails with the following type error:
    -- * Promotion should have occurred at: a1. Pawns must be promoted when they reach the opposite end of the board.
    -- * When checking the inferred type:
    --     didntPromoteBlack :: Data.Proxy.Proxy (TypeError ...)
didntPromoteBlack = create
        put _Wh _P at h7
        put _Bl _P at a2
    startMoves
        pawn h7 promoteTo _B h8
        pawn a2 to a1
    end
\end{lstlisting}

\end{overprint}
    
\end{frame}

\begin{frame}[fragile]{EDSL Game Testing}
    
Testing for EDSL correctness primarily consisted of writing out the first few moves of some famous game, and then making variations with errors:

\begin{overprint}

\onslide<1>\begin{lstlisting}
loopVsGandalf = chess
        p e4 p c5
        n f3 p d6
        p d4 p d4
        n d4 n f6
        n c3 p a6
    end
\end{lstlisting}

\onslide<2>\begin{lstlisting}
loopVsGandalfError = chess
        p e4 p c5
        n f3 p d6
        p d4 p d4
        n d4 n f6
        n c3 p a7
    end
\end{lstlisting}

\onslide<3>\begin{lstlisting}
loopVsGandalfError = chess
        p e4 p c5
        n f3 p d6
        p d4 p d4
        n d4 n f6
        n c3 p a7  -- Pawn moves to same place!
    end
\end{lstlisting}

\end{overprint}

\end{frame}

\begin{frame}{Compile-time and memory issues}

Compile-time and memory issues came up time and again throughout development; putting a hard limit on the length of Chesskell games.

With some games, GHC will run out of memory (\textgreater 25GB) and crash.

Through testing, it seems the upper limit is \textbf{12 moves maximum}; while all 6-move games tested have compiled, most 8- and 10-move games do as well.
    
\end{frame}

\begin{frame}[fragile]{Compile-time and memory issues cont.}

In fact, we discovered a difference in behaviour between type applications (\cite{typeapplication}) and type signatures:

\begin{lstlisting}
-- Compiles, but would hang
chess :: Spec (Proxy StartDec)
chess cont = cont (Proxy @StartDec)

-- Would fail to compile
chess :: Spec (Proxy StartDec)
chess cont = cont (Proxy :: Proxy StartDec)
\end{lstlisting}

\pause

So we filed a GHC bug report: \url{https://gitlab.haskell.org/ghc/ghc/-/issues/18902}
    
\end{frame}

\begin{frame}{Project Management}

The project was managed successfully, making use of;

\begin{itemize}
    \item<2-> Spiral methodology;
    \item<3-> Git and GitHub for version control;
    \item<4-> Weekly supervisor meetings;
    \item<5-> A Trello board to track upcoming tasks and completed features (Figure \ref{trello}); 
    \item<6-> Extensive unit and integration testing.
\end{itemize}

\begin{overprint}

\onslide<7>The implementation of Chesskell was feature-complete by the 4th of December; since then, work has gone into optimisation, testing, and write-ups.
    
\end{overprint}


% Ultimately, since there is no customer, an agile approach would have been a waste-so we used the \emph{spiral} methodology, moving from planning to implementation to internal testing, and repeating until done.

\end{frame}

\begin{frame}{Project Management cont.}
\begin{figure}
    \includegraphics[width=\linewidth]{trello.png}
    \caption{The Trello board used to track development.}
    \label{trello}
\end{figure}
\end{frame}

\begin{frame}{Further Work}

There is room to expand Chesskell:

\pause

\begin{itemize}
    \item<2-> A session-typed version of Chesskell;
    \item<3-> Further optimisations to try and increase the move limit;
    \item<4-> An automated tool to transform from Algebraic Notation into Chesskell notation.
\end{itemize}
    
\end{frame}

\begin{frame}{Conclusions}

We have created:

\pause

\begin{itemize}
    \item<2-> A full type-level model of Chess, which enforces all rules in the FIDE 2018 Laws of Chess;
    \item<3-> An EDSL for describing Chess games and creating custom chess boards, which uses the type-level model for rule-checking;
    \item<4-> A first draft of a Haskell Symposium paper about the development of Chesskell, including our findings on compile-time and memory usage issues.
\end{itemize}

\begin{overprint}
\onslide<5>Furthermore, Chesskell is unique and has never been done before. Though there is room for further work and improvement, Chesskell is a success!
\end{overprint}

    
\end{frame}

\begin{frame}[standout]

Q\&A
    
\end{frame}

\begin{frame}[allowframebreaks]{References}

\printbibliography
    
\end{frame}

\end{document}
