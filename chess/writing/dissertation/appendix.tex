\chapter{Appendix}

\begin{figure}
    \centering
    \fenboard{8/8/8/8/8/8/8/8 w - - 0 1}
    \showboard
    \caption{An empty chess board.}
    \label{chessboard}
\end{figure}

\begin{figure}
    \centering
    \newgame
    \showboard
    \caption{An standard chess board where all pieces are in their starting position.}
    \label{startboard}
\end{figure}

\begin{figure}
    \centering
    \fenboard{8/8/2Q5/8/4p3/8/8/8 w - - 0 1}
    \showboard
    \quad
    \hidemoves{1.Qe4}
    \showboard
    \caption{The White Queen captures a Black Pawn by moving to its position, and removing it from play.}
    \label{capture}
\end{figure}

\begin{figure}
    \centering
    \fenboard{8/8/8/8/8/4Q3/8/R3k3 w - - 0 1}
    \showboard
    \caption{The White Queen and White Rook place the Black King into checkmate.}
    \label{checkmate}
\end{figure}

\begin{figure}
    \centering
    \newgame
    \hidemoves{1.e4 e5 2. Nf3 Nc6 3.Bb5 a6}
    \showboard
    \caption{The position of the board after: 1.e4 e5 2. Nf3 Nc6 3.Bb5 a6}
    \label{algebraicexample}
\end{figure}

\begin{figure}
    \centering
    \fenboard{rnbqkbnr/pppppppp/8/8/4P3/8/PPPP1PPP/RNBQKBNR b - - 0 1}
    \showboard
    \caption{The board created with: (rnbqkbnr/pppppppp/8/8/4P3/8/PPPP1PPP/RNBQKBNR b KQkq e3 0 1)}
    \label{fenexample}
\end{figure}

\begin{figure}
    \centering
    \fenboard{8/8/3B4/8/1B6/8/8/8 w - - 0 1}
    \showboard
    \caption{A board where two White Bishops can move to c5.}
    \label{twobishops}
\end{figure}

\begin{figure}
    \centering
    \fenboard{r3k1pr/8/8/8/8/8/8/RP2K2R w - - 0 1}
    \showboard
    \caption{A board where the Black King can castle left but not right, and the White King can castle right but not left.}
    \label{castleboard}
\end{figure}